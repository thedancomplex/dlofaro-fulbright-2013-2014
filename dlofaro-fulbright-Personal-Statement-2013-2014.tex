\documentclass[12pt]{article}
\usepackage{times}
\usepackage[margin=1.0in]{geometry}
\pagestyle{empty}		% remove page numbering

\begin{document}

\begin{center}
Fulbright Statement of Grant Purpose: Personal Statement\\
Daniel M. Lofaro, Croatia - Mobile Manipulator Unmanned Aerial Vehicles for Mine Removal\\

\end{center}

\normalsize
Prior to 2008 I did not like to travel. 
I did not see the use in it, and to be honest I was scared.  
I had gone out of the country for family vacations before but I have never gone on my own.  
My foreign language skills at the time were medioker at best.  
I was afraid that if I went overseas to do research I would have trouble understanding people and them me.  
The is the story of an event that changed the course of my life and what made me believe that international travel and collaberation is key to personal and professional growth success.
Before I start the story I must inject that: 
(1) To put it lightly I have a love of all and anything that is robots, especially new and cutting edge, and will do almost anything to work with them.  
(2) Dr. Paul Oh, my advisor who I fully and completely trust, had just put me on the NSF-PIRE project where I am the primary caretaker and researcher on the cutting edge full-size humanoid robot Jaemi HUBO (a dream come true).  

In the fall of 2007 my Dr. Oh pushed me to apply for a National Science Foundation’s (NSF) East Asia and Pacific Summer Institute (EAPSI) fellowship.  
The NSF-EAPSI fellowship would allow me to travel to S. Korea and work in the HUBO Lab at the Korean Advanced Institute of Science and Technology (KAIST).  
In Dr. Oh's opinion this was a required part of my studies if I were to work with HUBO, thus because of my love of robotics and my position in the NSF-PIRE had to apply.
This is because at KAIST I would learn all there is to know about the HUBO robot.
In my mind there was one problem with applying to the NSF-EAPSI, I might get it.
This would mean I have to go to a dangerous place on the brink of war, S. Korea.
I was scared.  
I ended up receiving the fellowship and reluctantly traveled to S. Korea spent my entire summer of 2008.  
This was the summer that changed my life in the best of ways!  

%Becaues of my great desire to work with this extrordinary and much convincing by Dr. Oh I decided to apply.




	By the time I arrived in Seoul Intl. airport (Incheon Intl.) I had already come to the realization that I would not be returning to the U.S. until the coming fall.  
I would have to make the best of a situation that I "thought" was bad.  
I could not have been any more wrong.  
I did not realize it at first but as I toured S. Korea, met the locals and worked with the members of the HUBO Lab I came to the realization that I was the luckiest man in the world. 
 As the days from my arrival turned into weeks I had formed a close Korean friend base.
We would play basketball together, go to the bar and norebong (Korean karaoke) and talk about what apparently all young men talk about, girls.  
I learned that Koreans are not as different from American as I thought, except they were able to bring out my singing more than their American counterparts.  
Another important lesson that I learned is that the easest, and most fun, way for a guy learn a new language is to go to dinner with a girl.
It may sound weird but when you are trying to impress someone (a girl) you really put your mind to learning conversational Korean.  
Some of these friends were also labmates from HUBO Lab.
This made collaboration a breeze.  
Not only were my labmates very motivated and smart, our skill sets also complement each other.  
It ended up being the perfect match.  



By the end of my stint in S. Korea I have learned that the world is not as scary and dangerous as I thought.
It is a fun, interesting and beautiful place filled with rich culture that I just needed to explore.
The bubble that I lived in had been popped and now I am ready to take on the world.  
I proceeded to go back to S. Korea multiple times for collaboration work, conferences, invited talks and other events with my new found friends and collaberators.  
Because of my spectacular experience with S. Korea I sought out other research opportunities abraud.  
One pertinent example was working for the North Atlantic Treaty Organization (NATO).
I was the organizer for a two week workshop in Cesme, Turkey.  
While organizing this workshop I was able to correspond with researchers from all over the world.  
During the workshop in Cesme I became friends with Prof. Zdenko from University of Zagreb (he taught me Bela, a Croatian card game which I play with my friends in the U.S. to this day).  
From that relationship a member of Prof. Zdenko’s lab, Matko, came to my lab, the Drexel Autonomous Systems Lab (DASL) for a Fulbright in 2011-12.  
Matko and I have become very good friends.
We correspond regularly even now that his Fulbright stint is over.  
He volunteered to orginize all arrangements in Croatia for me, including housing, transportation etc.  What a good friend!
I want to go to Croatia for Fulbright not only because of the wonderful research experiences that are offered at the University of Zagreb, I also want to complete the cycle of researcher trading.  
That and I hear croatia is a beautiful place and I am sure I will get another award winning shot, and I just got an underwater enclusure for my Canon\texttrademark.

Being an amature photographer I could appceriate the beauty of S. Korea's  landscapes and cityscapes.
One of my pictures of urban Seoul even won first prize (\$500) in a photography contest. 

\end{document}


