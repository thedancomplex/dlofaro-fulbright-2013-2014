\documentclass[12pt]{article}
\usepackage{times}
\usepackage[margin=1.0in]{geometry}

\begin{document}

\begin{center}
Fulbright Statement of Grant Purpose: Personal Statement\\
Daniel M. Lofaro, Croatia - Mobile Manipulator Unmanned Aerial Vehicles for Mine Removal\\

\end{center}

\normalsize
Prior to 2008 I did not like to travel.  I did not see the use in it, and to be honest I was scared.  I had gone out of the country for family vacations before but I have never gone on my own.  My foreign language skills at the time were conversational at best.  I was afraid that if I went overseas to do research I would have trouble having people understand me.  In the fall of 2007 my advisor, Dr. Paul Oh who I fully and completely trust, pushed me to apply for a National Science Foundation’s East Asia and Pacific Summer Institute (NSF-EAPSI) fellowship.  Before I continue this story I must inject that: (1) to put it lightly I have a love of all and anything that is robots, especially new and cutting edge.  (2) Dr. Oh had just put me on the NSF-PIRE project where I get to be the primary caretaker and researcher on the cutting edge full-size humanoid robot Jaemi HUBO (a dream come true).  To continue, the NSF-EAPSI fellowship would allow me to travel to S. Korea and work in the HUBO Lab at the Korean Advanced Institute of Science and Technology (KAIST) and learn all there is to know about HUBO (the same robot we would be procuring through the NSF-PIRE in October 2008.  In my mind there was one problem with the plan of applying to the NSF-EAPSI, it would mean I have to go to a dangerous place on the brink of war S. Korea.  After much convincing by Dr. Oh he finally convinced me to apply and I did.  I ended up receiving the fellowship and reluctantly traveled to S. Korea spent my entire summer.  This was the summer that changed my life!  

	By the time I arrived in Seoul Intl. airport (Incheon Intl.) I had already come to the realization that I would not be returning to the U.S. until the coming fall.  I would have to make the best of a situation that I thought was bad.  I could not have been any more wrong.  I did not realize it at the time but as I toured around S. Korea, met the locals and worked with the member of the HUBO Lab I realized that I was the luckiest man in the world.  As the days from my arrival turned into weeks I had formed a close Korean friend base, we would play basketball together, go to the bar and norebong (like karaoke, litterely means “song room”) and talk about what apparently all young men talk about, girls.  I learned that Koreans are not as different from American as I thought, except they were able to bring out my singing more than their American counterparts.  Another important lesson that I learned is that the easest, and most fun, way for a guy learn a new language is to go to dinner with a girl, it may sound weird but when you are trying to impress someone you really put your mind to conversational Korean.  Having some of these friends also be labmates from HUBO Lab made collaboration a breeze.  Not only were my labmates very motivated and smart, our skill sets also greatly complement each other.  It ended up being the perfect match.  

	By the end of my stint in S. Korea I have learned that the world is not as scary and dangerous as I thought.  The landscape and cityscape are beautiful, one of my pictures of urban Seoul won first prize (\$500) in a photography contest (I am an amature photographer).  The bubble that I lived in had been popped and now I am ready to take on the world.  I proceeded to go back to S. Korea multiple times for collaboration work, conferences.  Because of my spectacular experience with S. Korea I also sought out other research travel opportunities.  One pertinent example was working for the North Atlantic Treaty Organization (NATO) being an organizer for a three week workshop in Cesme, Turkey.  While organizing this workshop I was able to correspond with researchers from all over the world.  During the workshop in Cesme I became friends with Prof. Zdenko from University of Zagreb (he taught me Bela, a Croatian card game, which i play with my friends in the U.S. to this date).  From that relationship a member of Prof. Zdenko’s lab, Matko, came to my lab, the Drexel Autonomous Systems Lab (DASL) for a Fulbright in 2011-12.  Matko and I have become very good friends and correspond regularly even now that his Fulbright stint is over.  He volunteered to take care of everything in Croatia, including housing, transportation etc.  After All what are friends for.  I would like to go to Croatia for Fulbright not only because of the wonderful research experiences that are offered at the University of Zagreb, I also want to complete the cycle of researcher trading.  That and I hear croatia is a beautiful place and I am sure I will get another award winning shot.

\end{document}
