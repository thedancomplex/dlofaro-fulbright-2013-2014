\documentclass[12pt]{article}
\usepackage{times}
\usepackage[margin=1.0in]{geometry}

\begin{document}

\begin{center}
\Large
Fulbright Statement of Grant Purpose: Personal Statement\\
\large
Daniel M. Lofaro, Croatia\\
Topic: Mobile Manipulator Unmanned Aerial Vehicle (MM-UAV) and Robotics\\
Title: Mobile Manipulator Unmanned Aerial Vehicles for Mine Removal\\

\end{center}

\normalsize
In 2009 I had one of the defining experiences of my life.  I brought my full-size humanoid robot, Jaemi Hubo, to the Please Touch Museum in Philadelphia.  I gave multiple hands on presentations over a three day period.  Each presentation was a hands on experience.  For participants I specifically chose children between the ages of 4 and 7 years of age to come on stage and help with the demonstration.  The priceless look on the children's faces as they were pushing the robot (to show how it is able to balance) and move as they shake its hand was a truly priceless moment that will stay with me forever and it a driving force to why I believe that getting robotics and the sciences out to the public is curtail to peak interest in the subject, especially to those who are typical not interested in any of the STEM (Science Technology Engineering Mathematics) fields.

I firmly believe in education.  The way I learn is through hands on activities and through real world examples.  This is one of the driving forces to why i bring the robot to public places and events.  it allows the public to learn and get excited about something that otherwise they might not have had the opportunity to work with in person.  In addition I have

I believe that traveling is a key requirement for proper personal and professional growth.  The world is getting smaller due to faster and cheaper communications methods.  The national economy is now being replaced by the world economy.  It is essential that I travel and gain a greater understanding of the world, i.e. the people that I will be working with.


My name is Daniel M. Lofaro, I am a Ph.D. Candidate in the Drexel Autonomous Systems Lab (DASL) at Drexel University.  I hold a Master's (MS) degree in Electrical Engineering and Control Systems and a undergraduate engineering degree (BS) in Electrical and Computer Engineering.  My current field of study is complex control systems and autonomy with a concentration in humanoid robotics.  I am the primary caretaker of the 4'3" tall (130cm) full-size humanoid robot Jaemi Hubo.  To date I have published 10+ journal and conference papers in the fields of control systems, humanoid robots, video/audio processing, human robot interaction and alternative teaching methods with robotics.  Additional publications are currently out for review.   My coworkers and friends describe me as dependable, hardworking and creative.  I hold a large network of professional collaborators in multiple universities in the U.S. as well as in multiple countries (S. Korea, Singapore, Croatia).  Networking is very important to me because it keeps doors open that might not otherwise be open for you.

In 2008 I became a National Science Foundation (NSF) fellow by receiving the NSF East Asia and Pacific Summer Institute (NSF-EAPSI) fellowship.  This allowed me to spend the summer of 2008 in Daejon S. Korea researching humanoid robotics at the Hubo Lab in the Korean Advanced Institute of Science and Technology (KAIST).

In 2009 my colleague Robert Ellengberg and I spent three days at the Philadelphia Please Touch Museum giving interactive presentations of our humanoid robot Hubo to children between the ages of 4 and 7.  In most cases the robot was larger than the children.  The event received great reviews from the children, parents as well as radio and TV stations.

In late 2009 I became the Technical Chair of the North Atlantic Treaty Organization Advanced Studies Institute (NATO-ASI).  This allowed me to spend three weeks in Cesme Tuekey at the NATO-ASI 2010 to run the five two week long workshops that I had spent the prior nine months organizing.  Participants of the workshops were from 23+ countries in the field of autonomous systems.  The overwhelming consensus was that the program was greatly successful as well as fun and interactive.  This culminated in a book publication on the contents of the event.

In July of 2010 I was invited to attend the Telluride Neuromorphic Cognition Engineering Workshop sponsored by the Institute of Neuromorphic Engineering (INE) and the NSF.  At the intensive three week workshop I created a visual recognition system based on a synthetic retina developed at ETH Zurich.  I also developed a concept for a multi-channel communications method using neuron timing.  At the end of the workshop I received special recognition for my work.

In December 2010 I created a workshop (along with DASL) to learn how to use our humanoid robot Jaemi Hubo at the international robotics conference Humanoids 2010.  Humanoids 2010 is the premier humanoid robot conference in the world and is sponsored by IEEE a world renowned electrical and computer engineering society (in which I have been a member since 2003).  At the workshop we successfully trained people to develop for the full-size humanoid robot Jaemi Hubo, using our custom software, in under two days.  Like other workshops this event received exceptional reviews.

In 2011 I became a Ph.D. candidate and shortly after proposed.  I am now working on the specific topic of "end-effector velocity control and stability for legged robots" also known as throwing.  In April of 2012 I made the Jaemi Hubo robot successfully threw the first pitch at a Major League Baseball game at Citizens Bank Park (Phillies vs. Cubs).  There were 45,196 spectators with thousands more watching on TV/ESPN.  This was important to me because my father is a huge baseball fan successfully completing this task was also a milestone in my work towards my dissertation.  

In 2012 we started a new collaboration with multiple universities in the U.S. (MIT, USC, Virginia Tech, Georgia Tech, Ohio State, and Perdue) as well as KAIST in S.Korea.  As a part of this project we procured more of the Hubo robots which we distributed to our partner universities to preform research in parallel on common platforms.  When this happens I go to the given university and train them on the robot.  In addition I am developing a common open source software platform (based on our work in Humanoids 2010) for use with the Hubo robots with an emphasis on ease of use and combinability/connectiability.

In late 2012 procured an invitation to demonstrate Hubo at Maker Faire 2012 in New York City.  This event is geared towards the DYI (do it yourself) group. (Event happens on the 28-30 of Sept I will say how it goes after that). 

In my academic past I have organized and have been invited to multiple international workshops (NATO-ASI, Humanoids 2012 and ASI), given many presentations/demonstrations to a variety of age groups (PTM, Maker Faire 2012), as well as had my dissertation work shown on the national stage (Pitch at the Major League baseball game).  All of the latter increases my professional network.  Fulbright is the next logical step to increasing my network and becoming a conduit to connect people/researchers from around the world (and all walks of life) together.


\end{document}
