\documentclass[12pt]{article}
\usepackage{times}
\usepackage[margin=1.0in]{geometry}

\begin{document}
\begin{center}
%\Large
Fulbright Statement of Grant Purpose: Project Proposal\\
%\large
Daniel M. Lofaro, Croatia - 
%Topic: Mobile Manipulator Unmanned Aerial Vehicle (MM-UAV) and Robotics\\
Mobile Manipulator Unmanned Aerial Vehicles for Mine Removal\\

\end{center}

%\normalsize
Unmanned aerial vehicles (UAV) are at the forfrunt of robotics research in the U.S. as well as in other countries such as Croatia.  The most recent examples of traditional UAVs (fixed-winged) are the U.S. military's Preditor and MAKO drones.  The University of Pensivania's (UPenn) General Robotics, Automation, Sensing and Perception (GRASP) Lab was amoung the first use the small, agile quadroptors (four rotor helicopters) to create fast, autonomous and percision flying UAVs.  All of the latter UAVs have no world manipulation capibilities (with the exception  of the Preditor drone when it shoots something with its mistle payload).  Adding a single degree of freedom (one joint) grasper (or manipulator) to the UAV to allow for easy payload pickup and drop off has been shown viabal through the efforst of Yale and UPenn (2010-2011).  Though these manipulators are able to attach themselves to an object they are fixed to the UAV and thus relying souly on the UAV to position the manipulator.  This is the current state of the art, however it does not allow for dectrus manipulation.  The next logical step is the creation of UAVs with the ability to manipulate the enviroment indipendent of overall movement of the craft (i.e. put multi joint arms on the UAV).   The affiliation between the Drexel Autonomous Systems Lab (DASL) at Drexel University in Philadelphia, PA (the lab in which I research) and the Laboratory for Robotics and Intelligent Control Systems (LARICS) at the University of Zagreb in Croatia has resulted in the only bleeding edge research with adding dectrus arms (multi joint) to a UAV for world manipulation.  DASL and LARICS has dubbed UAVs with dextrus arms Mobile Manipulating UAVs or MM-UAVs.  MM-UAVs are well suited for applications such as diffusing Improvised Explosive Devices (IED - U.S. priority), disarming land-minds (Croatian prority), collecting sample of materials, performing mantance on a bridges or buildings, or clearing rubble in hazerdous areas (Korpela 2012).  I propose to create a MM-UAV that has the dexterity to manipulate the environment via the use of attached  arms/manipulators.  A special focus will be put on dexterity for disarming IEDs and land-mines.  This will be done in conjunction with Dr. Kovacic Zdenko and Dr. Bogdan Stjepan at the LARICS.  This will show that MM-UAVs have the ability to perform dangerous, dexterous and delicate tasks that are typically proformed by humans.

%\section{Proposal}
Adding dectrus arms to a UAV creates many chalanges.  System stability (making sure it can fly), effect of dynamic inertal loads due to arms (what happens when the arms move and pickup/release thing), kinimatic design (how it is put together), software/comunications/sensors (what controls the UAV and how does it know that it is doing in the world) all need to be addressed.  The creation of this MM-UAV is considered a systems intergration problem.  I propose using parts I am already am skilled with, such as the GAUI\texttrademark 500X Quadcopter, Dynamixel\texttrademark actuators, my opensoruce control software I origonally desiged for a complex full-size humanoid robot running on the Linux opperating system and stability control algerithms from my Ph.D. dissertation spisifically designed for stability preforming dectrus manipulation (extremely simular to this task except on a legged robot).  Having great experence with the parts, software and algrothms allows for quick development, shorter time to maturity and greater probobility to bring the project to fruition.  This project is timely and compelling for the host country because MM-UAVs are in its infancies as a technology and this givens both Croatia and the U.S. a changce to get their countries noted as the first.  In addition LARICS is procuring a 3D printer/rapid prototyper (good for construction of the parts for the arms) and has multiple GAUI\texttrademark and other quadcopters avaliable for imediate use.  This will help in making my independent study work cone to throughition.

My proposed timeline is split into three distinct sections, platform/algerithm development, testing/verification, 



%\section{Proposal}
I propose on creating a MM-UAV system that is capable of flying in autonomous and semi-autonomous 
modes.  The MM-UAV will have decorous arms capable of manipulating the environment.  The arms 
will be attached to a rotocraft capable of hovering, either standard helicopter with tail prop, coaxial 
helicopter, or multi rotor helicopter (such as quadro or hex copters).  The specific helicopter will be 
denoted by the required payload and size needed.  Multiple rotocraft are available at LARICS for use 
with this project.  The brains of the machine will consist of a linux enabled micro controller allowing for 
a large amount of open source software to be run on it.  The arms will be made out of standard off the 
shelf servo motors and custom designed parts printed on the rapid prototyping machine available at 
the LARICS Lab.  This first stage will take approximately 3 months including the startup time in the lab.  
The next three months will be spent on modifying existing control algorithms to allow the mm-uav to 
maintain stable flight.  during this time multiple iterations of arm configuration and placement will be 
modeled tested and verified.  At the end of the second three months a final version of the hardware and 
control algorithms will be ready.  the final three months will be spent using the robot in simulated real 
world tests such as demeaning operations, IED removal etc.  due to the highly research oriented se on 
real land minds and IEDs will not be attempted during the fulbright stint.  In result there will be a 
minimum of two publications written on the topic, one in the hardware design of the system and one on 
the control system, as well as multiple videos available for public viewing.  as a part of my typical 
outreach I will have public demonstrations of the device in my ongoing attempt to increase 
exceptement about science and technology.  


%\section{Research Background}
My experence and education is ideally suited to the compleation of this project.  My masters thesus focused on control systems and my Ph.D. dissertation (in progress) focusses on stability of a robot during manipulation using dectrus arms.  My special skills and experence in controls allowed me to work with UAV as a consultant for U.S. UAV R\&D teams and ran UAV compeittions for graduate and undergraduate students in multiple universities from 2008-2011.  My controbutions in full-size humanoid robotics shows that I am proficent in the hardware, software and electronic portions of complex control systems.  MM-UAVs are considered complex control systems.
%Talk about how it is a new and blossiming field.  Idea comes from starwars probe droids (keep to about 6 sentences) Masters and Ph.D. dissertation focus on stability and verying load compensation.  Control systems for flying  robots is a  well known necessity for flying machines.  The addition of dextrus arms makes the system similar  to that of my dissertation in that it has arms that create inertial loads that you need to compensate for when  stabilizing the machine however the flying aspect creates new dynamics to look at.  The creation and control  of multiple legged and wheeled robot paltforms is a good stepping stone to this end of research.  



%\section{Intro}
MM-UAVs are the next logical step in the evolution of 
Talk about CROMAC and what the project is, i.e. flying vehicles with arms (keep to about 4 sentences)
Unmanned Aerial Vehicles, or UAVs, have become a common place in the world economy



%\section{Previous Collaberation}
work before with NATO-ASI on unmanned systems which lead to a fulbright student coming to our lab 
to work with us on UAVs with arms, now the next obvious step is for me to go and continue the work at 
their institution (4 sentences)

%\section{Language}
Through my experience with working directly with members of my host lab before I am confident that communication will not be an issue.  To further increase productivity I will take advantage of the Croatian language classes available at the University of Zagreb.

%\section{How will help the USA and Croatia}
helps the ongoing world effort to demine Croatia.  I have also helped orginize multiple conferences in 
the past and will work with Zdenko and Bogdan with orginizing other local/international conference, 
this will help becasue I am a native english speaker and engineering conferences are done in english. 
(3 sentences)

%\section{Host contries relation to topic}
LARICS works with quadrotor UAV's and has been workign with DASL to put arms on them since 2011 
(2 sentences)

%\section{what is my relation/experence}
eas statend in personal statement, I work with complex autonomous systems and stability.  I am 
extencively farmiliar with how the motion of massive objects such as arms effect the stability of a system 
(dissertation topic).  Masters in control systems  (4 sentences)

%\section{Misc.}
%\begin{itemize}
%\item it can be done in 8 to 10 months because it has been started
%\item i can do it with my current ability
%\item we are doing the pioneering work in this area.
%\end{itemize}

%\section{compelling and effects to USA}
explian how is compelling and that it brings our expertease of UAVs from the USA to Croatia for a 
nobile purpose of demining.  increases the ability of the USA to help the world and also creates 
technology that can be used for other jobs important to the USA such as IED disarming and purching 
survaliance (4 sentences)

%\section{how will help correer} 
this will help correer by increasing my network and collabrators (of which i already have many)  the 
stregthened relationship will allow for furthre work and open doors for post-dock position and future 
collaboration in accademia (my desired vocation).  (4 sentences)

%\section{social impact}
Outside of lab work i plan on exploring the culinary aspects of Croatia, both in fare and in spirits.  I have found that in my past travels you learn more about a culture when you eat with the locals.  This is exactly what I plan on doing.  In addition I am an amateur photographer and I do plan on documenting my entire visit through still media.

%\section{mutual understanding}
Fulfullment of this project strengthens the mutual understanding between the USA and Croatia 
usa gets the pride of helping others with the fixing an important problem and croatia helps push the 
technology for the usa in the bloomign field of MM-UAV (2 sentences)






\end{document}
