\documentclass[12pt]{article}
\usepackage{times}
\usepackage[margin=1.0in]{geometry}
\pagestyle{empty}		% remove page numbering
\linespread{0.92}
%\linespread{0.91}


\begin{document}

\begin{center}
\textbf{PERSONAL STATEMENT}\\
Daniel M. Lofaro, Croatia, Electrical and Computer Engineering in Robotics\\
%Project Title: Mobile Manipulator Unmanned Aerial Vehicles for IED and Mine Removal\\

\end{center}

\normalsize
Prior to 2008 I did not like to travel. 
I did not see the use in it, and to be honest I was scared or the world being a dangerous place.  
My foreign language skills at the time were mediocre at best.  
I was afraid that if I went overseas to do research I would have trouble understanding people and them me.  
The is the story of my life changed and why I believe that international travel and collaboration is key to personal and professional growth and success.
Before I start the story I must inject that: 
(1) To put it lightly I have a love of all and anything that is robots and will do almost anything to work with them.  
(2) My advisor Dr. Paul Oh, whom fully and completely trust, had just put me on the National Science Foundation (NSF) project grant entitled Project for International Research and Education (PIRE) where I am the primary caretaker full-size humanoid robot Jaemi HUBO (a dream come true).  

In the fall of 2007 my Dr. Oh pushed me to apply for a NSF East Asia and Pacific Summer Institute (EAPSI) fellowship.  
The NSF-EAPSI would allow me to travel to S. Korea and work in the HUBO Lab at the Korean Advanced Institute of Science and Technology (KAIST).  
In Dr. Oh's opinion this was a required part of my studies if I were to work with HUBO, thus because of my love of robotics and my position in the NSF-PIRE had to apply.
This is because at KAIST I would learn all there is to know about the HUBO robot.
In my mind there was one problem with applying to the NSF-EAPSI, I might get it.
This would mean I have to go to a dangerous place on the brink of war, S. Korea.
I was scared.  
I ended up receiving the fellowship and reluctantly travelled to S. Korea spent my entire summer of 2008.  
This was the summer that changed my life in the best of ways!  

%Becaues of my great desire to work with this extrordinary and much convincing by Dr. Oh I decided to apply.




	By the time I arrived in Seoul Intl. airport I had already come to the realization that I would not be returning to the U.S. until the coming fall.  
I would have to make the best of a situation that I ''thought`` was bad.  
As I toured S. Korea, met the locals and worked with the members of the HUBO Lab I came to the realization that I was the luckiest man in the world. 
 As the days from my arrival turned into weeks I had formed a close Korean friend base.
We would play basketball together, go to the bar and norebong (Korean karaoke) and talk about what apparently all young men talk about, girls.  
I learned that Koreans are not as different from American as I thought.
The friendships i formed made collaboration with HUBO Lab a breeze.  
In addition to my lab-mates being smart and motivated, our skill sets also complement each other.  
It ended up being the perfect match (also why I believe we are successful).  

The bubble that I lived in had been popped and now I am ready to take on the world.  
I still hold collaborations with my foreign colleagues to this day.
I actively seek out other research and business opportunities both domestic and abroad.
I have started two small businesses, one in consulting and one in smart phone app development.
I plan to start more with the people I will meet during my proposed Fulbright in Croatia.
I also actively sought out employment with the North Atlantic Treaty Organization (NATO) to organize  a two week workshop in Turkey.    
I specifically bring up this example because this is where I met my host for my proposed Fulbright in Croatia.
While organizing this workshop I was able to correspond with researchers from all over the world including Prof. Zdenko from University of Zagreb.
He taught me Bela, a Croatian card game which I play with my friends in the U.S. to this day, and we have been friends ever since.  
From that relationship a member of Prof. Zdenko lab, Matko, came to my lab for a Fulbright in 2011-12.  
Now the logical next step is to return the favor and go to Prof. Zdenko's lab for Fulbright.

%This proved to me yet again how important international collaberation is to me professionally as well as socially.

International travel and collaboration changed my life for the better.
Its importance professionally and socially are immeasurable.
When my bubble was popped and I joined the world, I learned and became a better man for it.
Fulbright will have a profound effect on me socially and also professionally, allowing me to increase my social network and start new business ventures. 


\end{document}


